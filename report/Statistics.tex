\chapter{Statistics}
\label{stat}
For the next four question we use data from Tradeview taken from coinbase going back to 2016 to the present day. After cleaning the data the head is a follows
\begin{verbatim}
			time	open	high	low	close	Volume	range
date_time							
2016-05-23 10:00:00	1463961600	13.91	13.91	13.61	13.61	0.78673	0.30
2016-05-24 10:00:00	1464048000	13.68	13.74	12.00	12.77	2753.23998	1.74
2016-05-25 10:00:00	1464134400	13.00	13.18	11.93	12.61	9697.18313	1.25
2016-05-26 10:00:00	1464220800	12.61	12.95	12.15	12.47	2989.89229	0.80
2016-05-27 10:00:00	1464307200	12.47	12.47	10.25	10.98	19334.80484	2.22
\end{verbatim}

\section{ What is the average daily range.}
Summary statistics for range.

\begin{verbatim}
count    1565.000000
mean       21.331188
std        34.627367
min         0.100000
25%         4.440000
50%        11.080000
75%        24.380000
max       433.570000
\end{verbatim}
The average daily range is 21.331188. Figure \ref{fig:range} shows the range distribution.
\begin{figure}[H]
\center
\includegraphics[width=0.9\textwidth]{fig/range.png}
\caption{Range distribution}
\label{fig:range}
\end{figure}

\section{ What is the average daily volume.}
Summary statistics for Volume.

\begin{verbatim}
count    1.565000e+03
mean     1.333461e+05
std      1.276997e+05
min      7.867300e-01
25%      5.343518e+04
50%      9.512464e+04
75%      1.669171e+05
max      1.322283e+06
\end{verbatim}
The average daily Volume is 1.333461e+05. Figure \ref{fig:vol_dist} shows the volume distribution.
\begin{figure}[H]
\center
\includegraphics[width=0.9\textwidth]{fig/vol_dist.png}
\caption{Volume distribution}
\label{fig:vol_dist}
\end{figure}
\section{ What would you define as a low volume days?}
We will define volume below the 25th percentile to have low volume. Thus, a day with volume below 5.343518e+04 is low.
\section{ What is the average weekend range.}
The data is grouped by day of the week. 0 = Monday,..., 6= Sunday. 

The summary statstics for the groups are

\begin{tabular}{lrrrrrrr}
\toprule
{} &        time &        open &        high &         low &       close &         Volume &      range \\
day &             &             &             &             &             &                &            \\
\midrule
0   &  1531396800 &  244.666652 &  254.059286 &  233.163839 &  244.640179 &  139691.779979 &  20.895446 \\
1   &  1531483200 &  244.642768 &  254.736205 &  231.990179 &  245.165446 &  145372.288791 &  22.746027 \\
2   &  1531569600 &  245.172812 &  255.199821 &  230.065089 &  243.928795 &  150107.028572 &  25.134732 \\
3   &  1531656000 &  243.915134 &  253.779531 &  231.644643 &  242.454085 &  151790.895260 &  22.134888 \\
4   &  1531440000 &  241.549215 &  251.391749 &  229.833072 &  243.245695 &  141879.322847 &  21.558677 \\
5   &  1531526400 &  243.247578 &  253.825830 &  236.150538 &  246.119417 &  100524.152772 &  17.675291 \\
6   &  1531612800 &  246.120852 &  254.312287 &  235.164081 &  245.684126 &  103816.814631 &  19.148206 \\
\bottomrule
\end{tabular}

Thus the average weekend range

$$\frac{17.675291+ 19.148206}{2} = 18.4117485$$

In addition we plot the boxplots of ranges per day give by Figure \ref{fig:rday}. There is no evidence to suggest weekends range more. 
\begin{figure}[H]
\center
\includegraphics[width=0.9\textwidth]{fig/rday.png}
\caption{Range box plots by day}
\label{fig:rday}
\end{figure}
\section{ What affect does an increase in open interest have on price?}
If open interest is rising it means the trend will continue.
\section{ What affect does a decrease in open interest have on price?}
if open interest is falling it means there will be a discontinuation of the trend.
\section{ Does any relationship exist between open interest and price?}
Open interest is a sentiment indicator.  It represents the total number of open contracts so it is only relevant in futures and options markets.  When traders get stopped out, take profits or get liquidated we would see a drop in open interest. When traders open new long and short positions we see a rise in open interest. 

In addition if there is a prolonged period of time where open interest is increasing, it is probable that volatility will increase.  This is because there are more traders that could be liquidated or stopped out. 

These patterns are best seen on the 1 hour time frame. 




\section{ What is the average session range and volume - asia euro usa.}


We define the asia, usa, euro session in UTC time.

\begin{verbatim}
exchange_times = {
    'asia': [time(),time(hour = 6)],
    'usa': [time(hour=13,minute=30),time(hour = 20)],
    'euro': [time(hour=8),time(hour=16,minute=30)],   
}
\end{verbatim}
The \texttt{time()} is the amount of time past $00:00$ UTC. For example \texttt{time(hour=13,minute=30)} is $13:30$ UTC.

We perform the following on a data set for the past 3 years from coinbase with an hourly resolution. Note that there were rate limits on the api so 88 separate http requests had to sent.

\begin{itemize}
\item data was categorised into asia, usa, euro sessions
\item For session it was resampled to daily data
\end{itemize}

This yielded the following statistics  for volume and range

\begin{verbatim}
For the asia session
             range        volume
count  1100.000000   1100.000000
mean      5.053318   7559.744723
std       7.499806   7136.945473
min       0.292857    798.873352
25%       1.608571   3196.329492
50%       2.829286   5258.516808
75%       5.315714   9015.883759
max      84.672857  55962.362899
----------------------------------------------------
For the usa session
             range        volume
count  1100.000000   1100.000000
mean      4.340860   4472.978179
std       6.641011   4877.610076
min       0.190000    407.690536
25%       1.266786   1813.502618
50%       2.240000   3046.332611
75%       4.462143   5280.163938
max      75.645714  62737.359162
----------------------------------------------------
For the euro session
             range         volume
count  1101.000000    1101.000000
mean      4.691296    6347.743122
std       7.187887    6717.636049
min       0.212222     692.604705
25%       1.443333    2548.551281
50%       2.537778    4424.716103
75%       4.984444    7611.190454
max     106.830000  102992.740244
\end{verbatim}

\section{ Work out the ATR of Eth in excel and read the ATR pdf.}
Here is the python code that computes the 14 period ATR. 
\begin{verbatim}
x = 14
df_ATR[f"atr_period_{x}"] = df_ATR["range"]
for i in range(1,x):
    df_ATR[f"atr_period_{x}"] = df_ATR[f"atr_period_{x}"] +  df_ATR[f"atr_period_{x}"].shift(i)
df_ATR[f"atr_period_{x}"]  = df_ATR[f"atr_period_{x}"]/df_ATR["open"].shift(x-1)

df_ATR.plot(figsize=(12,8))

plt.title("14 period ATR distribution")
plt.savefig("../../../report/fig/atr.png",dpi=250)
plt.show()
\end{verbatim}



\begin{figure}[H]
\center
\includegraphics[width=0.9\textwidth]{fig/atr.png}
\caption{14 period ATR for 3 years of hourly data on ETHUSD}
\label{fig:atr}
\end{figure}
Figure \ref{fig:atr} plots the ATR along with the range and open.


\section{ Work out the distribution of returns and read the pdf.}
The summary statistic for the returns distribution
\begin{verbatim}
count    26387.000000
mean         0.000090
std          0.011256
min         -0.196000
25%         -0.003853
50%          0.000060
75%          0.004015
max          0.183455
\end{verbatim}

\begin{figure}[H]
\center
\includegraphics[width=0.9\textwidth]{fig/ret.png}
\caption{Returns for 3 years of hourly data on ETHUSD}
\label{fig:ret}
\end{figure}


\section{ What is the most common time of day for price movements.}

Figure \ref{fig:pm} shows price movements by time of day. A higher variance means a higher price move. Thus, price moves the most at 2:00 UTC.
\begin{figure}
\center
\includegraphics[width=0.9\textwidth]{fig/pm.png}
\caption{Price movements by time of day for 3 years of hourly data on ETHUSD}
\label{fig:pm}
\end{figure}
 
\section{ What are the most common times with the most volume.}
Figure \ref{fig:vol_times} shows volume by time of day. Thus, most volume is at 2:00 and 3:00 UTC.
\begin{figure}
\center
\includegraphics[width=0.9\textwidth]{fig/vol_times.png}
\caption{Volume by time of day for 3 years of hourly data on ETHUSD}
\label{fig:vol_times}
\end{figure}

\section{ Is beginning of the month typically quieter then end of month?}
Figure \ref{fig:qt} shows volume by day of the month. Thus, most volume is in the middle of the month. The start and the end of the month have roughly the same volume.
\begin{figure}
\center
\includegraphics[width=0.9\textwidth]{fig/qt.png}
\caption{Volume by time of day for 3 years of hourly data on ETHUSD}
\label{fig:qt}
\end{figure}

\section{ List of days where it trades greater then its Standard deviation, check (1SD, 2SD, 3SD)}
Using z scores of the resampled daily return we have

\begin{verbatim}
For days between 1 and 2 SD
[Timestamp('2017-08-01 00:00:00'),
 Timestamp('2017-08-02 00:00:00'),
 Timestamp('2017-08-07 00:00:00'),
 Timestamp('2017-08-17 00:00:00'),
 Timestamp('2017-08-21 00:00:00'),
 Timestamp('2017-08-30 00:00:00'),
 Timestamp('2017-09-03 00:00:00'),
 Timestamp('2017-09-06 00:00:00'),
 Timestamp('2017-09-09 00:00:00'),
 Timestamp('2017-09-13 00:00:00'),
 Timestamp('2017-09-14 00:00:00'),
 Timestamp('2017-09-17 00:00:00'),
 Timestamp('2017-09-19 00:00:00'),
 Timestamp('2017-09-22 00:00:00'),
 Timestamp('2017-09-24 00:00:00'),
 Timestamp('2017-09-28 00:00:00'),
 Timestamp('2017-10-18 00:00:00'),
 Timestamp('2017-11-09 00:00:00'),
 Timestamp('2017-11-11 00:00:00'),
 Timestamp('2017-12-01 00:00:00'),
 Timestamp('2017-12-02 00:00:00'),
 Timestamp('2017-12-07 00:00:00'),
 Timestamp('2017-12-15 00:00:00'),
 Timestamp('2017-12-23 00:00:00'),
 Timestamp('2017-12-24 00:00:00'),
 Timestamp('2017-12-26 00:00:00'),
 Timestamp('2017-12-28 00:00:00'),
 Timestamp('2018-01-02 00:00:00'),
 Timestamp('2018-01-03 00:00:00'),
 Timestamp('2018-01-04 00:00:00'),
 Timestamp('2018-01-05 00:00:00'),
 Timestamp('2018-01-07 00:00:00'),
 Timestamp('2018-01-12 00:00:00'),
 Timestamp('2018-01-13 00:00:00'),
 Timestamp('2018-01-14 00:00:00'),
 Timestamp('2018-01-16 00:00:00'),
 Timestamp('2018-01-18 00:00:00'),
 Timestamp('2018-01-19 00:00:00'),
 Timestamp('2018-01-21 00:00:00'),
 Timestamp('2018-01-22 00:00:00'),
 Timestamp('2018-01-23 00:00:00'),
 Timestamp('2018-01-25 00:00:00'),
 Timestamp('2018-01-28 00:00:00'),
 Timestamp('2018-01-29 00:00:00'),
 Timestamp('2018-01-31 00:00:00'),
 Timestamp('2018-02-03 00:00:00'),
 Timestamp('2018-02-08 00:00:00'),
 Timestamp('2018-02-10 00:00:00'),
 Timestamp('2018-02-11 00:00:00'),
 Timestamp('2018-02-15 00:00:00'),
 Timestamp('2018-02-21 00:00:00'),
 Timestamp('2018-02-22 00:00:00'),
 Timestamp('2018-03-07 00:00:00'),
 Timestamp('2018-03-08 00:00:00'),
 Timestamp('2018-03-09 00:00:00'),
 Timestamp('2018-03-23 00:00:00'),
 Timestamp('2018-03-29 00:00:00'),
 Timestamp('2018-04-05 00:00:00'),
 Timestamp('2018-04-20 00:00:00'),
 Timestamp('2018-04-21 00:00:00'),
 Timestamp('2018-04-26 00:00:00'),
 Timestamp('2018-04-27 00:00:00'),
 Timestamp('2018-05-03 00:00:00'),
 Timestamp('2018-05-07 00:00:00'),
 Timestamp('2018-05-11 00:00:00'),
 Timestamp('2018-05-12 00:00:00'),
 Timestamp('2018-05-14 00:00:00'),
 Timestamp('2018-05-23 00:00:00'),
 Timestamp('2018-05-28 00:00:00'),
 Timestamp('2018-05-29 00:00:00'),
 Timestamp('2018-05-30 00:00:00'),
 Timestamp('2018-06-11 00:00:00'),
 Timestamp('2018-06-13 00:00:00'),
 Timestamp('2018-06-15 00:00:00'),
 Timestamp('2018-06-22 00:00:00'),
 Timestamp('2018-06-23 00:00:00'),
 Timestamp('2018-06-27 00:00:00'),
 Timestamp('2018-07-03 00:00:00'),
 Timestamp('2018-07-10 00:00:00'),
 Timestamp('2018-07-11 00:00:00'),
 Timestamp('2018-07-18 00:00:00'),
 Timestamp('2018-08-01 00:00:00'),
 Timestamp('2018-08-08 00:00:00'),
 Timestamp('2018-08-09 00:00:00'),
 Timestamp('2018-08-16 00:00:00'),
 Timestamp('2018-08-18 00:00:00'),
 Timestamp('2018-08-19 00:00:00'),
 Timestamp('2018-08-21 00:00:00'),
 Timestamp('2018-09-12 00:00:00'),
 Timestamp('2018-09-19 00:00:00'),
 Timestamp('2018-09-21 00:00:00'),
 Timestamp('2018-09-25 00:00:00'),
 Timestamp('2018-09-26 00:00:00'),
 Timestamp('2018-10-11 00:00:00'),
 Timestamp('2018-10-12 00:00:00'),
 Timestamp('2018-11-21 00:00:00'),
 Timestamp('2018-11-23 00:00:00'),
 Timestamp('2018-11-25 00:00:00'),
 Timestamp('2018-11-27 00:00:00'),
 Timestamp('2018-11-29 00:00:00'),
 Timestamp('2018-12-04 00:00:00'),
 Timestamp('2018-12-06 00:00:00'),
 Timestamp('2018-12-19 00:00:00'),
 Timestamp('2018-12-26 00:00:00'),
 Timestamp('2018-12-28 00:00:00'),
 Timestamp('2019-01-02 00:00:00'),
 Timestamp('2019-01-03 00:00:00'),
 Timestamp('2019-01-14 00:00:00'),
 Timestamp('2019-01-15 00:00:00'),
 Timestamp('2019-01-21 00:00:00'),
 Timestamp('2019-01-28 00:00:00'),
 Timestamp('2019-01-29 00:00:00'),
 Timestamp('2019-02-18 00:00:00'),
 Timestamp('2019-02-24 00:00:00'),
 Timestamp('2019-03-06 00:00:00'),
 Timestamp('2019-03-16 00:00:00'),
 Timestamp('2019-04-08 00:00:00'),
 Timestamp('2019-04-12 00:00:00'),
 Timestamp('2019-04-26 00:00:00'),
 Timestamp('2019-05-07 00:00:00'),
 Timestamp('2019-05-14 00:00:00'),
 Timestamp('2019-05-15 00:00:00'),
 Timestamp('2019-05-18 00:00:00'),
 Timestamp('2019-05-23 00:00:00'),
 Timestamp('2019-05-27 00:00:00'),
 Timestamp('2019-05-31 00:00:00'),
 Timestamp('2019-06-04 00:00:00'),
 Timestamp('2019-06-13 00:00:00'),
 Timestamp('2019-06-22 00:00:00'),
 Timestamp('2019-07-01 00:00:00'),
 Timestamp('2019-07-02 00:00:00'),
 Timestamp('2019-07-08 00:00:00'),
 Timestamp('2019-07-18 00:00:00'),
 Timestamp('2019-07-28 00:00:00'),
 Timestamp('2019-08-19 00:00:00'),
 Timestamp('2019-08-29 00:00:00'),
 Timestamp('2019-09-08 00:00:00'),
 Timestamp('2019-09-18 00:00:00'),
 Timestamp('2019-09-28 00:00:00'),
 Timestamp('2019-10-01 00:00:00'),
 Timestamp('2019-10-08 00:00:00'),
 Timestamp('2019-10-10 00:00:00'),
 Timestamp('2019-10-24 00:00:00'),
 Timestamp('2019-11-23 00:00:00'),
 Timestamp('2019-11-25 00:00:00'),
 Timestamp('2019-12-17 00:00:00'),
 Timestamp('2019-12-18 00:00:00'),
 Timestamp('2020-01-20 00:00:00'),
 Timestamp('2020-02-06 00:00:00'),
 Timestamp('2020-02-07 00:00:00'),
 Timestamp('2020-02-15 00:00:00'),
 Timestamp('2020-02-17 00:00:00'),
 Timestamp('2020-02-19 00:00:00'),
 Timestamp('2020-02-20 00:00:00'),
 Timestamp('2020-02-26 00:00:00'),
 Timestamp('2020-02-27 00:00:00'),
 Timestamp('2020-03-03 00:00:00'),
 Timestamp('2020-03-15 00:00:00'),
 Timestamp('2020-03-16 00:00:00'),
 Timestamp('2020-03-23 00:00:00'),
 Timestamp('2020-03-24 00:00:00'),
 Timestamp('2020-04-03 00:00:00'),
 Timestamp('2020-04-17 00:00:00'),
 Timestamp('2020-04-19 00:00:00'),
 Timestamp('2020-04-21 00:00:00'),
 Timestamp('2020-04-23 00:00:00'),
 Timestamp('2020-04-30 00:00:00'),
 Timestamp('2020-05-04 00:00:00'),
 Timestamp('2020-05-10 00:00:00'),
 Timestamp('2020-05-11 00:00:00'),
 Timestamp('2020-05-18 00:00:00'),
 Timestamp('2020-05-22 00:00:00'),
 Timestamp('2020-05-29 00:00:00'),
 Timestamp('2020-05-31 00:00:00'),
 Timestamp('2020-06-12 00:00:00'),
 Timestamp('2020-07-23 00:00:00'),
 Timestamp('2020-07-24 00:00:00'),
 Timestamp('2020-07-26 00:00:00'),
 Timestamp('2020-07-27 00:00:00'),
 Timestamp('2020-08-01 00:00:00'),
 Timestamp('2020-08-02 00:00:00')]
------------------------------------------------------------------------
For days between 2 and 3 SD
[Timestamp('2017-08-06 00:00:00'),
 Timestamp('2017-08-09 00:00:00'),
 Timestamp('2017-09-05 00:00:00'),
 Timestamp('2017-09-18 00:00:00'),
 Timestamp('2017-10-14 00:00:00'),
 Timestamp('2017-11-24 00:00:00'),
 Timestamp('2017-11-25 00:00:00'),
 Timestamp('2017-12-09 00:00:00'),
 Timestamp('2017-12-12 00:00:00'),
 Timestamp('2017-12-14 00:00:00'),
 Timestamp('2017-12-19 00:00:00'),
 Timestamp('2017-12-22 00:00:00'),
 Timestamp('2018-01-08 00:00:00'),
 Timestamp('2018-01-10 00:00:00'),
 Timestamp('2018-02-02 00:00:00'),
 Timestamp('2018-02-05 00:00:00'),
 Timestamp('2018-03-15 00:00:00'),
 Timestamp('2018-03-18 00:00:00'),
 Timestamp('2018-03-27 00:00:00'),
 Timestamp('2018-03-30 00:00:00'),
 Timestamp('2018-05-04 00:00:00'),
 Timestamp('2018-05-24 00:00:00'),
 Timestamp('2018-08-11 00:00:00'),
 Timestamp('2018-09-09 00:00:00'),
 Timestamp('2018-09-14 00:00:00'),
 Timestamp('2018-09-18 00:00:00'),
 Timestamp('2018-09-22 00:00:00'),
 Timestamp('2018-11-15 00:00:00'),
 Timestamp('2018-12-07 00:00:00'),
 Timestamp('2018-12-18 00:00:00'),
 Timestamp('2018-12-21 00:00:00'),
 Timestamp('2018-12-23 00:00:00'),
 Timestamp('2019-01-11 00:00:00'),
 Timestamp('2019-02-09 00:00:00'),
 Timestamp('2019-02-19 00:00:00'),
 Timestamp('2019-02-25 00:00:00'),
 Timestamp('2019-04-03 00:00:00'),
 Timestamp('2019-05-12 00:00:00'),
 Timestamp('2019-06-28 00:00:00'),
 Timestamp('2019-07-11 00:00:00'),
 Timestamp('2019-07-17 00:00:00'),
 Timestamp('2019-08-15 00:00:00'),
 Timestamp('2019-10-26 00:00:00'),
 Timestamp('2019-11-22 00:00:00'),
 Timestamp('2020-01-15 00:00:00'),
 Timestamp('2020-02-12 00:00:00'),
 Timestamp('2020-02-13 00:00:00'),
 Timestamp('2020-03-09 00:00:00'),
 Timestamp('2020-03-12 00:00:00')]
------------------------------------------------------------------------
For days greater than 3 SD
[Timestamp('2017-09-15 00:00:00'),
 Timestamp('2017-09-16 00:00:00'),
 Timestamp('2017-12-13 00:00:00'),
 Timestamp('2018-01-17 00:00:00'),
 Timestamp('2018-02-06 00:00:00'),
 Timestamp('2018-02-07 00:00:00'),
 Timestamp('2018-04-13 00:00:00'),
 Timestamp('2018-08-14 00:00:00'),
 Timestamp('2018-09-06 00:00:00'),
 Timestamp('2018-11-20 00:00:00'),
 Timestamp('2018-12-24 00:00:00'),
 Timestamp('2018-12-29 00:00:00'),
 Timestamp('2019-05-16 00:00:00'),
 Timestamp('2019-07-15 00:00:00'),
 Timestamp('2019-09-25 00:00:00'),
 Timestamp('2020-03-13 00:00:00'),
 Timestamp('2020-03-20 00:00:00'),
 Timestamp('2020-04-07 00:00:00')]
\end{verbatim}

\section{ Is the market more likely to go up or down?}
We will consider the daily resampled data for which the return are greater than 1SD (all element in the list above).
\begin{verbatim}
is_up
False    0.508065
True     0.491935
\end{verbatim}
So on moves greater than 1SD 51\% of the time the market moves down while 49\% of the time the market moves up.
\section{ How does the market move when it is $>5  \%  $ move in a day}
We consider the re sampled daily data. Figure \ref{fig:pctc} is the distribution the absolute percent change. We mark the 5 and 10 percent points.

\begin{figure}[H]
\center
\includegraphics[width=0.9\textwidth]{fig/pctc.png}
\caption{Absolute percent change for daily data}
\label{fig:pctc}
\end{figure}
 
We filter days by moves between 5 an 10 percent. For those days we have the following summary statistics.

\begin{tabular}{lrrrrrr}
\toprule
{} &          low &         high &         open &        close &        volume &     returns \\
\midrule
count &   160.000000 &   160.000000 &   160.000000 &   160.000000 &    160.000000 &  160.000000 \\
mean  &   393.254607 &   401.924226 &   397.955906 &   397.938922 &  10313.270548 &   -0.014387 \\
std   &   298.040808 &   305.717916 &   302.348791 &   302.372059 &   5178.683033 &    1.669869 \\
min   &    91.986250 &    93.544167 &    92.605833 &    92.745833 &   2646.986398 &   -7.903750 \\
25\%   &   182.481563 &   185.088958 &   183.830833 &   183.794271 &   6730.351709 &   -0.531250 \\
50\%   &   276.966875 &   282.016250 &   279.603958 &   280.097917 &   9324.437458 &    0.068333 \\
75\%   &   522.655104 &   533.031354 &   528.920313 &   527.474896 &  12478.833113 &    0.532396 \\
max   &  1347.235833 &  1377.132500 &  1366.006250 &  1363.753333 &  31125.085513 &    4.875833 \\
\bottomrule
\end{tabular}
\vspace{1cm}\\
We also calculate the most common day of the month and day of the week for a move between 5 and 10 percent.
\vspace{1cm}\\
\begin{tabular}{lrr}
\toprule
{} &  day\_of\_month &  day \\
\midrule
0 &          23.0 &  3.0 \\
\bottomrule
\end{tabular}
\section{ How does the market move when its $>10 \% $ in a day}
We repeat the same analysis but for moves greater than 10 percent.
\vspace{1cm}\\
\begin{tabular}{lrrrrrr}
\toprule
{} &          low &         high &         open &        close &        volume &    returns \\
\midrule
count &    41.000000 &    41.000000 &    41.000000 &    41.000000 &     41.000000 &  41.000000 \\
mean  &   352.536433 &   366.085457 &   359.722541 &   359.313923 &  19272.578006 &  -0.416596 \\
std   &   282.334042 &   295.841460 &   289.756459 &   288.996099 &  10841.346712 &   3.194639 \\
min   &    88.517083 &    91.353333 &    90.230417 &    89.629583 &   7716.548928 & -10.818750 \\
25\%   &   160.810833 &   164.762500 &   162.386667 &   162.990833 &  11806.177713 &  -1.111250 \\
50\%   &   224.170833 &   230.869583 &   228.202083 &   227.710417 &  16902.097708 &   0.010833 \\
75\%   &   534.257917 &   547.426667 &   542.708333 &   539.067500 &  22731.665617 &   0.795417 \\
max   &  1114.007083 &  1144.463750 &  1130.869167 &  1133.618333 &  66780.561746 &   7.758333 \\
\bottomrule
\end{tabular}
\vspace{1cm}\\
Notice that there's a big difference between the 75 percent and the hundred percent quantile of volume. This implies that as moves become more extreme the returns and the volume become more exponentially extreme. 

The most common day of the month and day of the week for a move greater than 10 percent.
\vspace{0.5cm}\\

\begin{tabular}{lrr}
\toprule
{} &  day\_of\_month &  day \\
\midrule
0 &          15.0 &  4.0 \\
\bottomrule
\end{tabular}
\section{ How does it move when its under $<5 \% $?}
We repeat the same analysis but for moves less than 5 percent.
\vspace{0.5cm}\\
\begin{tabular}{lrrrrrr}
\toprule
{} &          low &         high &         open &        close &        volume &     returns \\
\midrule
count &   899.000000 &   899.000000 &   899.000000 &   899.000000 &    899.000000 &  899.000000 \\
mean  &   293.647002 &   297.290324 &   295.549123 &   295.579857 &   4807.769736 &    0.031466 \\
std   &   201.763226 &   205.377567 &   203.713711 &   203.762793 &   3447.003381 &    0.687126 \\
min   &    82.847917 &    83.723750 &    83.414583 &    83.293333 &    725.863348 &   -5.000417 \\
25\%   &   168.759792 &   170.563542 &   169.801250 &   169.625000 &   2586.147062 &   -0.144891 \\
50\%   &   223.373750 &   225.083333 &   224.404167 &   224.436667 &   3916.392366 &    0.011250 \\
75\%   &   317.486875 &   321.048750 &   319.219583 &   319.338333 &   5954.465718 &    0.198125 \\
max   &  1311.549583 &  1339.132500 &  1328.027500 &  1329.035833 &  32320.300813 &    6.276250 \\
\bottomrule
\end{tabular}
\vspace{0.5cm}\\
The most common day of the month and day of the week for a move less than 5 percent.
\vspace{0.5cm}\\
\begin{tabular}{lrr}
\toprule
{} &  day\_of\_month &  day \\
\midrule
0 &           4.0 &  6.0 \\
\bottomrule
\end{tabular}\begin{tabular}{lrr}
\toprule
{} &  day\_of\_month &  day \\
\midrule
0 &           4.0 &  6.0 \\
\bottomrule
\end{tabular}
\section{ What are the days before and after like of both $>5 \% $ and $<5 \% $ and $> 10 \% $}
This will be analysed further in the technical section but in general big moves are preceded by small volume or ranging markets. Once a big move has occurred generally the market will go back to ranging.
For extraordinarily big moves there is usually a correction. For example in December 2017 the price shot up but around early 2018 it dumped back to the same level.
\section{ Does it tend to trend or range more?}
From Figure \ref{fig:pctc} we see that there is more mass in the area below 5\% so markets are ranging more often than trending.
\section{ Do stationary test}
We use the ADF test from the python stats model package. We seek a 95\% significance level.

\begin{verbatim}
p value for hourly close 0.37742176859100934 likely not stationary
p value for hourly returns 0.0 likely stationary
p value for hourly this year close 0.9879970913768994 likely not stationary
p value for hourly close from aug 1 0.4431708128652189 likely not stationary
\end{verbatim}

\section{If you used the POC as the fair value for the next trading day, how often does price come back to test this area? (Point of Control)}
We separate the data by days and use the following algorithm to find the POC of each day.

\begin{verbatim}
POC = np.zeros(len(list_by_date))
for i in range(len(list_by_date)):
    a,b = np.histogram(list_by_date[i]["close"],weights=list_by_date[i]["volume"])
    idx = a.argmax()
    POC[i] = (b[idx] + b[idx+1])/2
POC
\end{verbatim}
We find

\begin{verbatim}
68.9108910891089% of the time price comes back to the POC the next trading day
\end{verbatim}
\section{Does over average in volume generally relate to bigger price movements? Does this generally last for more then one day?}
Yes this is further dressed in the technical section. We will find that there is a high correlation between returns and volume. Volume spikes are short lived and last for 1 or 2 days.  After the spike, the price usually corrects itself or  ranges for the next week.
\section{Average transactions on the network per day}

Figure \ref{fig:trans} \cite{tr} show daily average transactions for several cryptos during Q2 of 2020. Last quarter ETH did 866400 transactions on average per day  .


\begin{figure}[H]
\center
\includegraphics[width=0.9\textwidth]{fig/trans.png}
\caption{Average daily transactions}
\label{fig:trans}
\end{figure}

\section{Does increase in transactions increase demand and price?}

The more transactions a cryptocurrency does the more people believe in the technology so it's likely that in the future the price will rise.  However for big sudden spikes in transactions it could be worrying.  For example the founder of Sushi swap cashed in a lot of Sushi for ETH which made the price fall quite dramatically.